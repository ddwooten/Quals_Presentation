\documentclass{beamer}

% These two are for the beamer class
\usepackage{fancyvrb}
\usepackage{color}

% This is for figures
\usepackage{graphicx}

%For tables
\usepackage{multirow}

% This is the path to the majority of the figures
\graphicspath{{/home/ddwooten/Dropbox/Dans_Stuff/Documents/UCB/Quals/Presentation/Figures/}}

% For equation formatting
\usepackage{amsmath}

% For sizing summations properly
\usepackage{relsize}

% Biblography Style
\bibliographystyle{elsarticle-num}

\definecolor{forestgreen(web)}{rgb}{0.13, 0.55, 0.13}
\definecolor{cardinal}{rgb}{0.77, 0.12, 0.23}
\mode<presentation>
{
  \definecolor{berkeleyblue}{HTML}{003262}
  \definecolor{berkeleygold}{HTML}{FDB515}
  \usetheme{Boadilla}      % or try Darmstadt, Madrid, Warsaw, ...
  %\usecolortheme{dove} % or try albatross, beaver, crane, ...
  \setbeamercolor{structure}{fg=berkeleyblue,bg=berkeleygold}
  \setbeamercolor{palette primary}{fg=berkeleyblue,bg=berkeleygold} % changed this
  \setbeamercolor{palette secondary}{bg=berkeleyblue,fg=white} % changed this
  \setbeamercolor{palette tertiary}{bg=berkeleyblue,fg=white} % changed this
  \usefonttheme{structurebold}  % or try serif, structurebold, ...
  \useinnertheme{circles}
  \setbeamertemplate{navigation symbols}{}
  \setbeamertemplate{caption}[numbered]
  \usebackgroundtemplate{}
}

\title[Modelling of Material Transfers in Reactor Fuel Burnup Simulations]{}
\author{Daniel D. Wooten}
\institute{UC Berkeley}
\date{October 20\textsuperscript{th}, 2017}

\begin{document}

\frame{\titlepage}

\begin{frame}

\frametitle{Outline}

    \begin{itemize}
        \item Molten salt reactors ( MSR ) 
        \item Fuel salt evaluation 
        \item ADER
        \item Keeping the problem linear
        \item The optimal solution
        \item Modifying the Bateman equations
        \item Designing for parallel operation
        \item Verification and Vallidation
        \item Future Work
        \item References
    \end{itemize}

\end{frame}

\begin{frame}
\frametitle{What is an MSR?}

    \begin{figure}
        \centering
        \includegraphics[width=0.5\paperwidth, keepaspectratio]{MSR}
        \caption{Figure coutesey of Idaho National Labs}
        \label{fig:msr_pic}
    \end{figure}

\end{frame}

\begin{frame}
\frametitle{Fuel Salt Selection}

Considerations \ldots

    \begin{itemize}
        \item Structural materials corrosion 
        \item Heavy element solubility
        \item Isotopic evolution
    \end{itemize}

\end{frame}

\begin{frame}
\frametitle{The question}

How does the fuel salt composition in a molten salt fueled reactor change over
time in response to both nuclear fuel burnup and reactor operator actions?

\end{frame}

\begin{frame}
\frametitle{Finding the answer}

The only general purpose tool available, Monteburns, only addresses isotopic
evolution.
\\
For better answers we need better tools.

\end{frame}

\begin{frame}
\frametitle{Pertinent phenomenon}

    \begin{enumerate}
        \item Solubility
        \item Reduction - Oxidation ( Redox ) potential of the salt
        \item Reactor reactivity
        \item Available chemical forms
    \end{enumerate}

\end{frame}

\begin{frame}
\frametitle{Advanced Dynamic Depletion Extension for Reprocessing - ADDER}

A source code modification to the SERPENT 2 Monte Carlo code which enables the
user to \ldots

    \begin{enumerate}
        \item define material transfers into, out of, and between SERPENT materials
        \item define and enforce relationships between isotopes, elements, and chemicals in SERPENT materials
        \item define system k-eigenvalue targets
        \item define a weighted-average target for the valance state of a given SERPENT material
    \end{enumerate}

\end{frame}

\begin{frame}
\frametitle{Best practices and ADDER}

    \begin{enumerate}
        \item Input
        \item Output
        \item Failures
        \item Stlye
        \item Architecture
        \item Testing
        \item Documentation
    \end{enumerate}

\end{frame}

\begin{frame}
\frametitle{In conclusion\ldots}

    Questions?

\end{frame}

\begin{frame}
\frametitle{Acknowledgements}

    This material is based upon work supported under a Department of Energy,
    Office of Nuclear Energy, Integrated University Program Graduate Fellowship.

\end{frame}

\begin{frame}[allowframebreaks]
\frametitle{References}

%\bibliography{Quals}

\end{frame}

\begin{frame}
\frametitle{Corrosion by fuel salts}

    \begin{itemize}
        \item 316 Stainless steel, in a 316L stainless steel crucible, in FLiNaK (46.5\%LiF - 11.5\%NaF - 42\%KF) salt has a reported corrosion rate of 17.1$\mu$m/yr. \cite{zheng_corrosion_2015}
        \item 316 Stainless steel, in a 316L stainless steel crucible, in FLiBe salt has a reported corrosion rate of 125$\mu$m/yr $\pm$12.5$\mu$m/yr.
    \end{itemize}

\end{frame}

\begin{frame}
\frametitle{Heavy element solubility}

    \begin{figure}
        \centering
        \includegraphics[width=0.5\paperwidth, keepaspectratio]{LiF_UF4_ThF4_Liquidus}
        \caption{Figure 5.2 from \cite{rosenthal_development_1972}}
        \label{fig:lifufthf_liquidus}
    \end{figure}

\end{frame}

\end{document}
