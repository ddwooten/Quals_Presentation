\documentclass{beamer}

% These two are for the beamer class
\usepackage{fancyvrb}
\usepackage{color}

% This is for figures
\usepackage{graphicx}

%For tables
\usepackage{multirow}

% This is the path to the majority of the figures
\graphicspath{{/home/ddwooten/Dropbox/Dans_Stuff/Documents/UCB/Quals/Presentation/Figures}}

% For equation formatting
\usepackage{amsmath}

% For sizing summations properly
\usepackage{relsize}

% Biblography Style
\bibliographystyle{elsarticle-num}

\definecolor{forestgreen(web)}{rgb}{0.13, 0.55, 0.13}
\definecolor{cardinal}{rgb}{0.77, 0.12, 0.23}
\mode<presentation>
{
  \definecolor{berkeleyblue}{HTML}{003262}
  \definecolor{berkeleygold}{HTML}{FDB515}
  \usetheme{Boadilla}      % or try Darmstadt, Madrid, Warsaw, ...
  %\usecolortheme{dove} % or try albatross, beaver, crane, ...
  \setbeamercolor{structure}{fg=berkeleyblue,bg=berkeleygold}
  \setbeamercolor{palette primary}{fg=berkeleyblue,bg=berkeleygold} % changed this
  \setbeamercolor{palette secondary}{bg=berkeleyblue,fg=white} % changed this
  \setbeamercolor{palette tertiary}{bg=berkeleyblue,fg=white} % changed this
  \usefonttheme{structurebold}  % or try serif, structurebold, ...
  \useinnertheme{circles}
  \setbeamertemplate{navigation symbols}{}
  \setbeamertemplate{caption}[numbered]
  \usebackgroundtemplate{}
}

\title[CFR Kinetics Modelling]{Future Needs of Circulating Fuel Reactor Kinetics Modelling}
\author{Daniel D. Wooten}
\institute{UC Berkeley}
\date{May 22\textsuperscript{nd}, 2017}

\begin{document}

\frame{\titlepage}

\begin{frame}

\frametitle{Outline}

    \begin{itemize}
        \item What are Circulating Fuel Reactors ( CFRs )?
        \item Effects of circulating fuel
        \item Current kinetics models
        \item Key assumptions and their effects
        \item Future direction for CFR kinetics modelling
    \end{itemize}

\end{frame}

\begin{frame}
\frametitle{What is a CFR?}

    \begin{figure}
        \centering
        \includegraphics[width=0.5\paperwidth, keepaspectratio]{MSR}
        \caption{Figure coutesey of Idaho National Labs}
        \label{fig:msr_pic}
    \end{figure}

\end{frame}

\begin{frame}
\frametitle{Effects of circulating fuel - Delayed Neutron Precursors ( DNPs )}

    \begin{columns}

        \column{0.5\linewidth}

        \begin{figure}
            \centering
            \includegraphics[width=0.45\paperwidth, keepaspectratio]{Dulla_Models_2005_Fig_1_9}
            \caption{Figure 1.9 from \cite{dulla_models_2005}, showing 
            simulated DNP concentrations in the MSRE.}
            \label{fig:dulla_models_c}
        \end{figure}

        \column{0.5\paperwidth}

        \begin{figure}
            \centering
            \includegraphics[width=0.45\paperwidth, keepaspectratio]{Dulla_Models_2005_Fig_2_4}
            \caption{Figure 2.4 from \cite{dulla_models_2005}, showing simulated
            DNP importance values in the MSRE.}
            \label{fig:dulla_models_c_star}
        \end{figure}

    \end{columns}

\end{frame}

\begin{frame}
\frametitle{Effects of circulating fuel - $\beta_{eff}$}
\begin{table}[h]
    \caption{Value of $\beta_{eff}^{flow}/\beta_{eff}^{static}$ given out of
        core residence time, $\tau_{l}$, in seconds, 
        and fluid flow velocity, $v_{fluid}$ in [cm/s]. Data taken from
        the last two rows of table 5
        in \cite{mattioda_effective_2000}. A 1D reactor model with height of 3 m
        modeled while a 3 group neutron diffusion approach is used to generate
        the given data.} 
    \label{tbl:mattioda_beta_reduction}
    \begin{center}
        \begin{tabular}{|c|c|c|c|c|}
            \hline
            $\tau_{l}\rightarrow$ & 0 & 5 & 10 & 15 \\
            \hline
            $v_{fluid} = 60$ & 0.843 & 0.539 & 0.468 & 0.440 \\
            \hline
            $v_{fluid} = 100$ & 0.834 & 0.420 & 0.352 & 0.329 \\
            \hline
        \end{tabular}
    \end{center}
\end{table}
\end{frame}

\begin{frame}
\frametitle{Effects of circulating fuel - neutron flux}
    
    \begin{figure}
        \centering
        \includegraphics[width=0.5\paperwidth, keepaspectratio]{Zhang_Development_2009-1_Fig_15_A}
        \caption{Fast (1) and thermal (2) neutron fluxes in a hypothetical MSR of height 3.8 m, diameter 3.4 m,
    radial reflector thickness of 0.5 m, core power of 2.4 GWth, inflow temperature of 873 K,
    0.5 $\frac{m}{s}$ fluid flow
    velocity and a 10 s out-of-core residence time.Reproduced from Figure 15.a from
            \cite{zhang_development_2009-1}. 
               Axial
               values are taken at the center of the core.}
        \label{fig:zhang_development_flux}
    \end{figure}

\end{frame}

\begin{frame}
\frametitle{Current kinetics models - Challenges}

    \begin{itemize}
        \item DNPs are displaced from their point of origin
        \item Many plausible transients induce strong spatial effects on the
                flux
    \end{itemize}

\end{frame}

\begin{frame}
\frametitle{Current kinetics models - Options}

    \begin{itemize}
        \item Transport based methods - primarily Multi-Group neutron Diffusion
                ( MGD )
        \item Point-Reactor Kinetics ( PRK ) based methods
        \item Quasi-Statics ( QS ) based methods
    \end{itemize}

\end{frame}

\begin{frame}
\frametitle{Current kinetics models - MGD}

\begin{equation}
\label{eq:mgd}
\begin{split}
\frac{1}{v_{g}} \cdot \frac{\partial \phi_{g}}{\partial t} &+ \nabla \cdot
    (\phi_{g} \overset{\rightharpoonup}u) = \nabla \cdot D_{g} \nabla \phi_{g}
    + \mathlarger{\sum}_{g \prime \not= g} \Sigma_{s,g \prime \rightarrow g}
    \phi_{g \prime} - \mathlarger{\sum}_{g \prime \not= g} 
    \Sigma_{s,g \rightarrow g \prime} \phi_{g} \\
    &+ (1-\beta)\chi_{p,g}
    \mathlarger{\sum}_{g \prime = 1}^{G}\frac{1}{k_{eff}}
    (\nu \Sigma_{f})_{g \prime}\phi_{g\prime} + 
    \mathlarger{\sum}_{i = 1}^{I}\chi_{d,g}\lambda_{i}C_{i} 
     - \Sigma_{a,g}\phi_{g}
\end{split}
\end{equation}

\begin{equation}
\label{eq:mgd_dnp}
\nabla \cdot (C_{i} \overset{\rightharpoonup}u) - 
    \nabla \cdot \frac{\nu_{T}}{Sc_{T}}\nabla C_{i}
    = \beta_{i} \mathlarger{\sum}_{g = 1}^{G} \frac{1}{k_{eff}} 
    (\nu \Sigma_{f})_{g} \phi_{g} - \lambda_{i} C_{i}
\end{equation}

\end{frame}

\begin{frame}
\frametitle{Current kinetics models - MGD performance}

    \begin{figure}
        \centering
        \includegraphics[width=0.5\paperwidth, keepaspectratio]{Krepel_DYN3D_2007_Fig_9}
   \caption{Figure 9 from \cite{krepel_dyn3d-msr_2007} showing both the experimental MSRE data
   and the simulated data from the DYN1D and DYN3D codes, using both the JEF and ORNL DNP data, for
   the MSRE pump start transient with ``Reactivity" referring to the change in reactivity introduced
   by the control rod motion used to keep the MSRE at a steady state power level.} 
   \label{fig:krepel_dyn3d_msre_pump_start}
        \label{fig:krepel_dyn3d_mgd}
    \end{figure}

\end{frame}

\begin{frame}
\frametitle{Current kinetics models - PRK Variants}

    \begin{itemize}
        \item Delayed Point reactor Kinetics ( DPK )
        \item Iterative Point reactor Kinetics ( IPK )
        \item Modified Point reactor Kinetics ( MPK )
    \end{itemize}

\end{frame}

\begin{frame}
\frametitle{Current kinetics models - PRK}
\begin{equation}
    \label{eq:prk}
    \frac{dn(t)}{dt} = \frac{\rho(t) - \beta_{eff}}{\Lambda} n(t) +
     \sum_{i = 1}^{I} \lambda_{i} C_{i}(t)
\end{equation}

\begin{equation}
    \label{eq:prk_dnp}
    \frac{dC_{i}(t)}{dt} = \frac{\beta_{eff}^{i}}{\Lambda} n(t) -
        \lambda_{i} C_{i}(t)
\end{equation}

\end{frame}

\begin{frame}
\frametitle{Current kinetics models - DPK}
\begin{equation}
    \label{eq:dpk_dnp}
    \frac{dC_{i}(t)}{dt} = \frac{\beta_{eff}^{i}}{\Lambda} n(t) -
        \lambda_{i} C_{i}(t) - \frac{C_{i}(t)}{\tau_{c}} +
        \frac{C_{i}(t - \tau_{l}) e^{-\lambda_{i} \tau_{l}}}{\tau_{c}}
\end{equation}

\end{frame}

\begin{frame}
\frametitle{Current kinetics models - PRK/DPK performance}

    \begin{figure}
        \centering
        \includegraphics[width=0.5\paperwidth, keepaspectratio]{Zhang_Comparison_2009_Fig_1}
        \caption{Figure 1 from \cite{zhang_comparison_2009} showing relative power (dashed line) and flow (solid lines) for a 
   ULOF transient in the MOSART reactor as simulated using differing kinetics methods.}
        \label{fig:zhang_comparison_ulof}
    \end{figure}

\end{frame}

\begin{frame}
\frametitle{Current kinetics models - PRK/DPK performance}

    \begin{columns}

        \column{0.5\linewidth}

    \begin{figure}
        \centering
        \includegraphics[width=0.4\paperwidth, keepaspectratio]{Zhang_Comparison_2009_Fig_2}
        \caption{Figure 2 from \cite{zhang_comparison_2009} showing average fuel salt temperature for a
   ULOF transient in the MOSART reactor as simulated using differing kinetics methods.}
        \label{fig:zhang_comparison_temp}
    \end{figure}

        \column{0.5\linewidth}

    \begin{figure}
        \centering
        \includegraphics[width=0.4\paperwidth, keepaspectratio]{Zhang_Comparison_2009_Fig_3}
        \caption{Figure 3 from \cite{zhang_comparison_2009} showing reactivity contributions for a
   ULOF transient in the MOSART reactor as simulated using differing kinetics methods.}
        \label{fig:zhang_comparison_reac}
    \end{figure}

    \end{columns}

\end{frame}

\begin{frame}
\frametitle{Current kinetics models - DPK performance}

    \begin{figure}
        \centering
        \includegraphics[width=0.5\paperwidth, keepaspectratio]{Shi_Development_2016_Fig_3}
        \caption{Figure 3 from \cite{shi_development_2016} showing 
    compensative control rod reactivity inserted during the MSRE pump start
   transient; both measured, and simulated with a modified RELAP5 code.}
        \label{fig:shi_development_relap}
    \end{figure}

\end{frame}

\begin{frame}
\frametitle{Current kinetics models - IPK performance}

    \begin{figure}
        \centering
        \includegraphics[width=0.5\paperwidth, keepaspectratio]{Merle_Lucotte_Physical_2015_Fig_3_Top}
        \caption{Figure 3 (top) from \cite{merle-lucotte_physical_2015} showing the simulated
   MSFR response to an instantaneous change in the extracted power.}
        \label{fig:merle_physical_reac}
    \end{figure}

\end{frame}

\begin{frame}
\frametitle{Current kinetics models - MPK performance}

    \begin{figure}
        \centering
        \includegraphics[width=0.5\paperwidth, keepaspectratio]{Dulla_Models_2005_Fig_2_17}
        \caption{Figure 2.17 from \cite{dulla_models_2005} showing the simulated response of the MSRE
   to a 4x increase in its flow velocity as modeled by differing methods.}
        \label{fig:dulla_models_mpk}
    \end{figure}

\end{frame}

\begin{frame}
\frametitle{Current kinetics models - Quasi-Statics}

    \begin{figure}
        \centering
        \includegraphics[width=0.5\paperwidth, keepaspectratio]{Dulla_Models_2005_Fig_2_18}
        \caption{Figure 2.18 from \cite{dulla_models_2005} showing different results of the
   quasi-static method when simulating a 4x increase in the MSRE fluid flow velocity using
   differing numbers of shape recalculations.}
        \label{fig:dulla_models_qs}
    \end{figure}

\end{frame}

\begin{frame}
\frametitle{Assumptions}

    \begin{itemize}
        \item Turbulent diffusion
        \item Flow regime
        \item Secondary loop inclusion
    \end{itemize}

\end{frame}

\begin{frame}
\frametitle{Turbulent diffusion}

\begin{figure}[H]
   \centering
   \includegraphics[width=0.5\textwidth]{Cheng_Development_2014_Fig_19}
   \caption{Figure 19 from \cite{cheng_development_2014} showing the DNP
   concentration in a
   hypothetical MSR core with thermal power of 2.4 GW, inflow fluid flow
   velocity of 0.5
   $\frac{m}{s}$, and residence time out-of-core of 3.94 s.
   The left image shows the fourth
   DNP group concentration in-core without any fluid flow effects. The center
   image
   includes advective flow, and the right image includes both advective flow and
   turbulent diffusion.} 
   \label{fig:cheng_diffusion}
\end{figure}

\end{frame}

\begin{frame}
\frametitle{Turbulent diffusion}

\begin{figure}[H]
   \centering
   \includegraphics[width=0.5\textwidth]{Aufiero_Calculating_2014_Fig_9}
   \caption{Figure 9 from \cite{aufiero_calculating_2014} showing the effect of varying
   the turbulent Schmidt number on the calculation of $\beta_{eff}$ in the MSFR.}
   \label{fig:aufiero_sc}
\end{figure}

\end{frame}

\begin{frame}
\frametitle{Flow regime}

\begin{figure}[H]
   \centering
   \includegraphics[width=0.5\textwidth]{Dulla_Interactions_2007_Fig_12}
   \caption{Figure 12 from \cite{dulla_interactions_2007} showing the relative power in a
   simulated MSRE transient involving a 2x increase in fluid flow velocity with time profile
   shown in the upper right corner. Please refer to
   Figures \ref{fig:dulla_flow_map} and \ref{fig:dulla_flow_adf} for an
   explanation of the flow regimes seen here. In this figure, circles correspond
   to a ``slug" (uniform) flow regime, asterisks to a ``parabolic" flow regime,
   stars to an
   ``ADF" flow regime, squares to an ``SOD" flow regime, and triangles to an
   ``SOI" flow regime.} 
   \label{fig:dulla_flow_regimes}
\end{figure}

\end{frame}

\begin{frame}
\frametitle{Flow regime}

\begin{table}[h]
    \caption{Reproduced from Table III of \cite{dulla_interactions_2007}. 
    ``Slug"
        refers to a uniform velocity profile across the reactor core.
        The other flow regimes seen in this table are explained in
        Figures \ref{fig:dulla_flow_map} and \ref{fig:dulla_flow_adf}. This
        table shows the reactivity change in a simulated MSRE corresponding to
        different fluid fuel flow patterns. The reactivity change is given both
        overall and broken down into contributions from each DNP group.} 
    \label{tbl:dulla_flow_regimes_beta}
    \begin{center}
	\resizebox{\linewidth}{0.25\paperheight}{
        \begin{tabular}{|c|c|c|c|c|c|c|c|c|c|}
            \hline
            \multirow{2}{*}{Flow Regime} & \multirow{2}{*}{$\Delta \rho$(pcm)}
            &\multirow{2}{*}{$\lambda(s^{-1})$}&1 & 2 & 3 & 4 & 5 & 6 &
             \multirow{2}{*}{Total} \\
            \cline{4 - 9}
            & & & 0.0127 & 0.0317 & 0.116 & 0.311 & 1.4 & 3.87 & \\
            \hline
            No flow & 0 & $\beta$ (pcm) & 23.7 & 122.7 & 117.1 & 262.7 & 107.9 &
                45.1 & 679.2 \\
            \hline
            Slug & -251.5 & $\beta_{eff}^{i}$ (pcm) & 7.5 & 41.8 & 52.7 & 178.2&
                 103.5 & 44.9 & 428.6 \\
            & & $\beta_{eff}^{i}/\beta^{i}$ & 0.318 & 0.341 & 0.450 & 0.678 &
                0.959 & 0.995 & 0.631 \\
            \hline
            Parabolic & -299.6 & $\beta_{eff}^{i}$ (pcm) & 7.4 & 39.6 & 44.2 &
                146.5 & 98.8 & 44.5 & 381.0 \\
            & & $\beta_{eff}^{i}/\beta^{i}$ & 0.313 & 0.323 & 0.378 & 0.558 &
                0.916 & 0.986 & 0.561 \\
            \hline
            ADF & -310.4 & $\beta_{eff}^{i}$ (pcm) & 5.8 & 33.5 & 41.9 & 145.8 &
                 98.7 & 44.5 & 370.2 \\
            & & $\beta_{eff}^{i}/\beta^{i}$ & 0.246 & 0.273 & 0.358 & 0.555 &
                0.915 & 0.986 & 0.545 \\
            \hline
            SOD & -281.2 & $\beta_{eff}^{i}$ (pcm) & 8.0 & 40.9 & 47.0 & 157.9 &
                100.8 & 44.7 & 399.3 \\
            & & $\beta_{eff}^{i}/\beta^{i}$ & 0.336 & 0.334 & 0.402 & 0.601 &
                0.934 & 0.990 & 0.588 \\
            \hline
            SOI & -251.5 & $\beta_{eff}^{i}$ (pcm) & 7.5 & 42.9 & 56.6 & 188.9 &
                104.4 & 45.0 & 445.3 \\
            & & $\beta_{eff}^{i}/\beta^{i}$ & 0.315 & 0.350 & 0.483 & 0.719 &
                0.968 & 0.996 & 0.655 \\
            \hline
        \end{tabular}}
    \end{center}
\end{table}

\end{frame}

\begin{frame}
\frametitle{Secondary loop inclusion}

    \begin{columns}

        \column{0.5\linewidth}

\begin{figure}[H]
   \centering
   \includegraphics[width=0.8\textwidth]{Zanetti_Development_2016_Fig_2_5_A}
   \caption{Experimental and
   simulated results of a 24 pcm reactivity insertion in the MSRE during
    1 MWth operation, as reproduced from Figure 7a in 
    \cite{zanetti_extension_2015}}
   \label{fig:zanetti_sec_1mw}
\end{figure}

        \column{0.5\linewidth}

\begin{figure}[H]
   \centering
   \includegraphics[width=0.8\textwidth]{Zanetti_Development_2016_Fig_2_6_A}
   \caption{Experimental and
   simulated results of a 19 pcm reactivity insertion in the MSRE during 5 MWth
    operation, as reproduced from Figure 6a in \cite{zanetti_extension_2015}}
   \label{fig:zanetti_sec_5mw}
\end{figure}

\end{columns}

\end{frame}

\begin{frame}
\frametitle{Future Modelling  - Challenges}
    
    \begin{itemize}
        \item Academia and industry both rely heavily on parametric design and
            safety analysis
        \item Within known and constrained operational envelopes PRK based
            methods perform quite well
        \item The nature of many parametric analyses makes the use of PRK based
            methods questionable
        \item MGD based methods are computational expensive
    \end{itemize}

\end{frame}

\begin{frame}
\frametitle{Future Modelling  - Proposed solution}
    
    It should be evident that a high quality, user-friendly, time-adaptive, and
    general-purpose quasi-statics code be developed fur use by all parties
    interested in the kinetics modelling of CFRs. This code should ideally
    \ldots

    \begin{itemize}
        \item Reactor agnostic
        \item Multiple PRK implementations
        \item Open source
        \item Written in a modern object-oriented coding language ( C++ )
        \item Be well documented
        \item Verified against existing CFR kinetics codes
        \item Validated against availble ( MSRE ) data
    \end{itemize}

\end{frame}

\begin{frame}
\frametitle{In conclusion\ldots}

    Questions?

\end{frame}

\begin{frame}
\frametitle{Acknowledgements}

    This material is based upon work supported under a Department of Energy,
    Office of Nuclear Energy, Integrated University Program Graduate Fellowship.

\end{frame}

\begin{frame}[allowframebreaks]
\frametitle{References}

\bibliography{Kinetics}

\end{frame}

\begin{frame}
\frametitle{Flow Regime Map 1}

\begin{figure}[H]
   \centering
   \includegraphics[width=0.6\textwidth]{Dulla_Interactions_2007_Fig_1_A}
   \caption{Figure 1.a from \cite{dulla_interactions_2007} showing the radial profile of
   differing flow regimes used in the cited study. Circles correspond
    to a ``parabolic" flow regime, stars to a ``step constant outward
     decreasing"
    (SOD) flow regime, and squares to a ``step constant outward increasing"
     (SOI)
    flow regime.} 
   \label{fig:dulla_flow_map}
\end{figure}

\end{frame}

\begin{frame}
\frametitle{Flow Regime Map 2}

\begin{figure}[H]
   \centering
   \includegraphics[width=0.6\textwidth]{Dulla_Interactions_2007_Fig_1_B}
   \caption{Figure 1.b from \cite{dulla_interactions_2007}.
   This figure depicts
   the radial profile of flow velocity for several reactor core heights
   corresponding to an ``axially developing field" (ADF) flow regime.} 
   \label{fig:dulla_flow_adf}
\end{figure}

\end{frame}

\end{document}
